\section{Previous Work} \label{work}

Previous work on and with EvoSuite has been done to prove automated test case generation could result in test suites with high code coverage and little test cases \cite{FRASER2013}.
The same team who presented EvoSuite additionally inquired upon the effectiveness of EvoSuite and the \textit{whole-test-suite} approach \cite{FRASER2014}. 

\subsection{Making of EvoSuite}

Making EvoSuite with the \textit{whole-test-suite} approach aimed at high code coverage and the generation of smaller test suites to make the specification of assertions as easy as possible \cite{FRASER2013}.

EvoSuite achieves this by selecting branch coverage as a coverage criterion and executing a genetic algorithm that selects a random population of test suites and evolves this population.
It stops the evolution process when a solution is found which fulfills the coverage criterion or when the set of resources (such as time or number of fitness evaluations) have been used up \cite{FRASER2013}. 

A solution is defined as a \textit{test suite}, which is represented as a set of test cases. 
A test case is represented by a sequence of statements. 
The fitness function estimates how close a solution is to covering \textit{all} branches by taking into account \textit{branch distance} for each branch, and the number of branches left to cover. 
It also takes into account the size of the solution, to make sure the generated test suite does not grow to large. 

It was shown that this approach fulfilled large coverage goals and succeeded at generating small enough test suites for easy oracle generation \cite{FRASER2013}. 
But what about the performance of the \textit{whole-test-suite} approach compared to \textit{single-target}?

\subsection{Whole Test Suite vs Single Target}

By comparing the coverage achieved by the \textit{whole-test-suite} and \textit{single-target} approach, it was establish that, in general, the \textit{whole} approach performs better \cite{FRASER2014,FRASER2015}.

Existing research has shown that the \textit{whole-test-suite} approach lead to better coverage results than the traditional \textit{single-target} approach. 
There was reasonable doubt on whether this would be the case when targeting difficult goals. 
An in-depth analysis was performed to study if this doubt could be confirmed in practice \cite{FRASER2014}.

EvoSuite was run on 100 different Java classes respectively for both techniques. 
It was found that indeed, the Whole approach had more difficulties when targeting difficult goals. 
However, these cases are very few compared to the cases for which Whole performed significantly better.
This thus confirmed that overall, \textit{whole-test-suite} is better. 



