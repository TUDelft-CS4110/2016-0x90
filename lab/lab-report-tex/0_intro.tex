\section{Introduction} 

Software testing is an essential component of software development that allows developers to assess the reliability and correctness of a software product.
Over the years, researchers have designed techniques to automatically produce test cases, where the code coverage criterion is the most commonly used guiding medium for the generation. 
Now, with the newest tools available, it has become possible to generate entire test suites with high coverage. 
One of these tools for Java software is the EvoSuite  framework \cite{FRASER2013}.

\subsection{EvoSuite} \label{evosuite}

EvoSuite is a tool that automatically generates JUnit tests with an evolutionary algorithm \cite{JUNIT,FRASER2013}. 
This tool can be directly run from the command line or through the Eclipse or IntelliJ IDEA IDEs, with the options to either generate tests according to the \textit{whole-test-suite} or \textit{single-target} approach \cite{FRASER2013}. 
It can also be configured to either generate tests class per class or for the entire project at once \cite{EVOSUITECLI}.

EvoSuite works at the bytecode level and collects all necessary information for test case generation from the bytecode via Java Reflection \cite{FRASER2013}. 
This means that EvoSuite can also be used for other programming languages that compile to Java bytecode (such as Scala or Groovy) \cite{VENNERS2008,SUBRAMANIAM2008}. 

There is, however, one problem associated with automatic test case generation, known as the \textit{oracle problem} \cite{BARR2015}. 
Automatic oracles that assert what the outcome of a test should be are possible when it comes to explicit events such as 'the program should not crash'. 
But when dealing with more complex outcomes, human intervention is necessary to specify oracles \cite{FRASER2013}. 
For this to be feasible, EvoSuite aims not only at high code coverage but also at small test suites that make manual oracle generation as easy as possible. 

\subsection{Limitations of EvoSuite}

But EvoSuite does not always get it right, and some coverage goals are occasionally not fulfilled \cite{FRASER2013}. 
This can be explained by a few language-related problems that are encountered by the EvoSuite tool during test case generation.

In particular, classes using Java Generics are problematic, as type erasure removes much of the useful information during compilation and all generic parameters are considered as Object. 
To overcome this problem, EvoSuite always inserts an Integer object into container classes, in order to cast returned Object instances back to Integer \cite{FRASER2013}. 
However, for a method with Generic input that is not compatible with an Integer object, things can go wrong. 

Additionally, other factors may contribute to explaining why EvoSuite does not always get it right.
There may be methods with inputs for which conscious human intervention is necessary, and there is no way for EvoSuite to know what the input is supposed to be \cite{FRASER2013}. 
An instance of this problem is a method that takes an integer and returns \textit{true} if that integer is prime. For a test to cover the \textit{true} branch and check the correctness of the code, the input must be a prime number. 

\subsection{This Research}

This paper poses the question of difficulty behind automated test case generation with EvoSuite and dives into the reasons why such difficulty exists. 
Why are certain pieces of code hard to generate tests for? 
Does EvoSuite fail at generating tests because the code is bad or too complex?
Does EvoSuite always generate readable tests to ease the generation of oracles by developers?

After a recap of previous work done on and with EvoSuite in section \ref{work}, the methods for this research are explained in detail in section \ref{methods}.
Then, the results are presented in section \ref{results}.
Lastly, this paper concludes the research executed and brings up a discussion on the implication of the results found. 

On a side note, an analysis of the usability of the EvoSuite tool is given. 
This analysis provides a few details on the problems encountered when using EvoSuite and what difficulties regarding the usage of the framework possibly hindered the progress of this research. 
